% Greek Abstract (Περίληψη)

\begingroup
\selectlanguage{greek}

\chapter*{Περίληψη}
\addcontentsline{toc}{chapter}{Περίληψη}
\thispagestyle{plain}

Η νόσος του Πάρκινσον (ΝΠ) είναι μια νευροεκφυλιστική διαταραχή που χαρακτηρίζεται από κινητικά συμπτώματα και εκτεταμένες διαταραχές ομιλίας. Η παρούσα διπλωματική εργασία διερευνά τη δυνατότητα ανίχνευσης της ΝΠ μέσω φωνητικής ανάλυσης, χρησιμοποιώντας μεθόδους κλασικής μηχανικής μάθησης, με έμφαση στη μεθοδολογική αυστηρότητα έναντι της μέγιστης απόδοσης. Εξετάζονται δύο συμπληρωματικά σύνολα δεδομένων: το Σύνολο Δεδομένων Α, μια κλινική συλλογή ακατέργαστων ηχογραφήσεων φωνής (37 υποκείμενα) που απαιτεί εξαγωγή ακουστικών χαρακτηριστικών, και το Σύνολο Δεδομένων Β, ένα μεγαλύτερο δημόσιο σύνολο (756 δείγματα) προ-εξαγμένων χαρακτηριστικών.

Εφαρμόζεται ένας ενιαίος αγωγός επεξεργασίας, εξάγοντας 47 βασικά χαρακτηριστικά (προσωδιακά μέτρα και μέτρα διαταραχής) από το Σύνολο Δεδομένων Α, καθώς και ένα εκτεταμένο σύνολο 78 χαρακτηριστικών. Τρεις ερμηνεύσιμοι ταξινομητές---Λογιστική Παλινδρόμηση, Μηχανή Διανυσμάτων Υποστήριξης (πυρήνας RBF) και Τυχαίο Δάσος---αξιολογούνται υπό σχέδιο 2$\times$2: βασικά έναντι εκτεταμένων χαρακτηριστικών, με και χωρίς στάθμιση κλάσεων. Χρησιμοποιείται ομαδοποιημένη διαστρωματοποιημένη 5-πτυχη διασταυρούμενη επικύρωση για το Σύνολο Δεδομένων Α ώστε να αποφευχθεί η διαρροή δεδομένων.

Τα αποτελέσματα αναφέρονται ως μέσος όρος $\pm$ τυπική απόκλιση. Στο Σύνολο Δεδομένων Α, το καλύτερο μοντέλο (Τυχαίο Δάσος, εκτεταμένα χαρακτηριστικά) πέτυχε ROC-AUC $\approx$ 0,87 $\pm$ 0,14. Τα εκτεταμένα χαρακτηριστικά βελτίωσαν σταθερά την ακρίβεια, ιδιαίτερα για το μικρότερο Σύνολο Δεδομένων Α. Το Σύνολο Δεδομένων Β παρουσίασε υψηλότερη απόλυτη απόδοση (ROC-AUC $\approx$ 0,94) αλλά ερμηνεύεται με επιφύλαξη. Συμπερασματικά, τα κλασικά μοντέλα μηχανικής μάθησης μπορούν να ανιχνεύσουν τη ΝΠ μέσω φωνής με ανταγωνιστική ακρίβεια, αλλά η αυστηρή επικύρωση είναι καθοριστική.

\vspace{1cm}

\textbf{Λέξεις Κλειδιά:} Νόσος του Πάρκινσον, Δυσαρθρία, Φωνητικοί Βιοδείκτες, Ακουστικά Χαρακτηριστικά, Μηχανική Μάθηση, Εξαγωγή Χαρακτηριστικών, Διασταυρούμενη Επικύρωση, Μη Ισορροπημένα Δεδομένα, Αναπαραγωγιμότητα, Ταξινόμηση Ομιλίας, Υπολογιστική Νοημοσύνη στην Υγεία

\endgroup
\blankpage
