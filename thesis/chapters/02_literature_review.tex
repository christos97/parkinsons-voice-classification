% Chapter 2: Literature Review

\chapter{Literature Review}
\label{ch:literature-review}

\section{Parkinson's Disease and Speech Impairment}

Parkinson's disease (PD) is a progressive neurodegenerative disorder primarily known for its motor symptoms such as tremor, rigidity, and bradykinesia. In addition to these well-recognized motor features, PD almost invariably affects speech and voice as the disease progresses. Studies report that approximately 70--90\% of individuals with PD develop measurable speech and voice impairments. This collection of speech symptoms in PD is often referred to as \textit{hypokinetic dysarthria}, indicating a characteristic pattern of speech motor control impairment associated with the disease.

The speech of a person with PD typically exhibits several hallmark changes:

\begin{itemize}
    \item \textbf{Hypophonia} --- reduced voice loudness; patients often speak in a much softer voice than normal
    \item \textbf{Monotonic pitch} --- limited range of pitch, resulting in speech that lacks normal intonation
    \item \textbf{Monoloudness} --- little variation in volume
    \item \textbf{Articulatory imprecision} --- consonants not enunciated crisply
    \item \textbf{Breathy or hoarse voice quality} --- reflecting incomplete vocal fold closure
\end{itemize}

Crucially, speech changes in PD are of interest not just as symptoms affecting communication, but also as potential non-invasive biomarkers of the disease. Voice is relatively easy to capture (e.g., via a short recording on a phone), and vocal changes can manifest early in the disease course. Some research suggests that subtle voice abnormalities may appear even before classic motor symptoms in some patients.

\section{Acoustic Characteristics of Parkinsonian Speech}

\subsection{Prosodic Features}

Prosodic features relate to the pitch (fundamental frequency) and loudness (intensity) patterns in speech. The fundamental frequency of the voice (notated as $F_0$) corresponds to the perceived pitch. Prosodic analysis often examines:

\begin{itemize}
    \item Mean pitch and pitch range
    \item Standard deviation of pitch across an utterance
    \item Overall intensity level and its variability
\end{itemize}

In Parkinson's disease, a well-documented phenomenon is the reduction of prosodic variability. PD patients often speak in a monotone---their pitch remains relatively flat and at a narrow range. Objectively, one finds a lower standard deviation of $F_0$ and a smaller pitch range in PD speech compared to healthy age-matched controls.

\subsection{Perturbation Measures}

Perturbation measures capture the cycle-to-cycle variations in the voice signal, reflecting stability (or instability) of vocal fold vibration. The two primary perturbation measures are:

\begin{description}
    \item[Jitter] Quantifies minute fluctuations in pitch period from one glottal cycle to the next
    \item[Shimmer] Quantifies variability in amplitude across successive glottal cycles
\end{description}

In Parkinson's disease, due to factors like reduced vocal fold adduction, tremor, and inconsistent breath support, jitter and shimmer are often elevated compared to age-matched healthy controls.

\textbf{Harmonics-to-Noise Ratio (HNR)} is another related metric, comparing the level of periodic (harmonic) energy in the voice to aperiodic or noise energy. PD voices tend to have lower HNR values, indicating a higher proportion of noise (breathiness, roughness) in the voice.

\subsection{Spectral and Cepstral Features}

The disease also manifests in spectral characteristics:

\begin{itemize}
    \item Altered formant frequencies ($F_1$, $F_2$, $F_3$)
    \item Changes in Mel-Frequency Cepstral Coefficients (MFCCs)
    \item Modified spectral envelope characteristics
\end{itemize}

\section{Feature Extraction Approaches}

\subsection{Traditional Acoustic Features}

Early studies relied on clinically-motivated features:

\begin{table}[H]
\centering
\begin{tabular}{@{}llp{6cm}@{}}
\toprule
\textbf{Category} & \textbf{Examples} & \textbf{Physiological Basis} \\
\midrule
Fundamental Frequency & $F_0$ mean, $F_0$ std & Vocal fold tension \\
Perturbation & Jitter, Shimmer & Neuromuscular control \\
Noise & HNR, NHR & Incomplete glottal closure \\
Formants & $F_1$, $F_2$, $F_3$ & Vocal tract configuration \\
\bottomrule
\end{tabular}
\caption{Traditional acoustic feature categories}
\label{tab:traditional-features}
\end{table}

\subsection{Spectral Features}

Modern approaches incorporate signal processing features:

\begin{itemize}
    \item \textbf{MFCCs} (Mel-Frequency Cepstral Coefficients) --- compact spectral representation
    \item \textbf{Delta and Delta-Delta MFCCs} --- temporal dynamics
    \item \textbf{Spectral shape features} --- centroid, bandwidth, rolloff, flatness
\end{itemize}

\subsection{Deep Learning Features}

Recent work has explored end-to-end learning from spectrograms. However, these approaches require large datasets and lack interpretability---both significant limitations for clinical applications.

\section{Machine Learning Approaches}

\subsection{Classical Methods}

Classical ML remains dominant in clinical applications due to interpretability:

\begin{table}[H]
\centering
\begin{tabular}{@{}lp{4.5cm}p{5cm}@{}}
\toprule
\textbf{Method} & \textbf{Strengths} & \textbf{Limitations} \\
\midrule
Logistic Regression & Interpretable coefficients & Linear decision boundary \\
SVM & Effective in high dimensions & Kernel selection critical \\
Random Forest & Handles non-linearity, feature importance & Less interpretable than linear models \\
\bottomrule
\end{tabular}
\caption{Comparison of classical ML methods}
\label{tab:ml-methods}
\end{table}

\subsection{Deep Learning}

CNNs and RNNs have been applied to PD detection but face challenges:
\begin{itemize}
    \item Require large labeled datasets
    \item Prone to overfitting on small samples
    \item Limited interpretability for clinical validation
\end{itemize}

\section{Methodological Concerns in Literature}

\subsection{Data Leakage}

Many published studies fail to account for subject identity when splitting data:

\begin{quote}
``When multiple recordings exist per subject, random train/test splits can place recordings from the same subject in both sets, leading to optimistic performance estimates.''
\end{quote}

This thesis addresses this through \textbf{grouped stratified cross-validation}.

\subsection{Class Imbalance}

Imbalanced class distributions are common but often unaddressed:
\begin{itemize}
    \item Simple accuracy can be misleading
    \item Class weighting or resampling strategies needed
    \item This thesis investigates \texttt{class\_weight="balanced"} as a mitigation
\end{itemize}

\subsection{Reproducibility}

Many studies lack sufficient detail for reproduction:
\begin{itemize}
    \item Feature extraction parameters unspecified
    \item Random seeds not fixed
    \item Cross-validation strategy unclear
\end{itemize}

\section{Research Gap}

While numerous studies report high classification accuracies, few address:

\begin{enumerate}
    \item \textbf{Grouped cross-validation} for multi-recording datasets
    \item \textbf{Controlled feature ablation} studies
    \item \textbf{Systematic class weighting} analysis
    \item \textbf{Transparent limitations} acknowledgment
\end{enumerate}

This thesis aims to fill these gaps through rigorous experimental design prioritizing methodological validity over performance optimization.

\section{Summary}

The literature demonstrates that voice-based PD detection is feasible, with classical ML achieving competitive results. However, methodological rigor varies significantly across studies. This thesis adopts a conservative approach, prioritizing reproducibility and valid comparison over state-of-the-art claims.
