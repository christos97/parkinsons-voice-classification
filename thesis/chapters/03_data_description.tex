% Chapter 3: Data Description

\chapter{Data Description}
\label{ch:data-description}

\section{Overview}

This thesis utilizes two distinct datasets for Parkinson's Disease voice classification:

\begin{table}[H]
\centering
\begin{tabular}{@{}lll@{}}
\toprule
\textbf{Property} & \textbf{Dataset A (MDVR-KCL)} & \textbf{Dataset B (PD\_SPEECH)} \\
\midrule
Data Type & Raw audio (WAV) & Pre-extracted features (CSV) \\
Source & Zenodo & Kaggle \\
Unit of Analysis & Subject (multiple recordings) & Sample (row) \\
Subject IDs Available & Yes & No \\
Total Samples & 73 recordings (37 subjects) & 756 samples \\
\bottomrule
\end{tabular}
\caption{Dataset comparison summary}
\label{tab:dataset-comparison}
\end{table}

\section{Dataset A: MDVR-KCL}

\subsection{Source and Collection}

The Mobile Device Voice Recordings from King's College London (MDVR-KCL) dataset was collected for PD research using smartphone recordings. Available on Zenodo with DOI: \texttt{10.5281/zenodo.2867215}.

\subsection{Speech Tasks}

The dataset includes two distinct speech tasks:

\begin{table}[H]
\centering
\begin{tabular}{@{}llccc@{}}
\toprule
\textbf{Task} & \textbf{Description} & \textbf{Subjects} & \textbf{HC} & \textbf{PD} \\
\midrule
ReadText & Reading a standardized passage & 37 & 21 & 16 \\
SpontaneousDialogue & Free conversation & 36 & 21 & 15 \\
\bottomrule
\end{tabular}
\caption{Speech tasks in MDVR-KCL dataset}
\label{tab:speech-tasks}
\end{table}

\textbf{Note:} Subject ID18 is missing from the SpontaneousDialogue task.

\subsection{Class Distribution}

\begin{verbatim}
ReadText Task:
+-- HC (Healthy Control): 21 subjects (56.8%)
+-- PD (Parkinson's Disease): 16 subjects (43.2%)

SpontaneousDialogue Task:
+-- HC (Healthy Control): 21 subjects (58.3%)
+-- PD (Parkinson's Disease): 15 subjects (41.7%)
\end{verbatim}

\textbf{Imbalance Ratio:} Moderate ($\sim$57:43), addressed via class weighting experiments.

\subsection{File Structure}

\begin{verbatim}
DATASET_MDVR_KCL/
+-- ReadText/
|   +-- HC/
|   |   +-- IDxx_hc_*.wav
|   +-- PD/
|       +-- IDxx_pd_*.wav
+-- SpontaneousDialogue/
    +-- HC/
    +-- PD/
\end{verbatim}

\subsection{Known Anomalies}

\begin{itemize}
    \item \textbf{ID22:} Non-standard filename pattern (handled in parsing code)
    \item \textbf{ID18:} Missing from SpontaneousDialogue task
    \item Multiple recordings per subject (requires grouped CV)
\end{itemize}

\subsection{Feature Correlation Analysis}


\begin{figure}[H]
    \centering
    \includegraphics[width=0.9\textwidth]{fig_heatmap_readtext.png}
    \caption{Feature correlation heatmap (ReadText task)}
    \label{fig:heatmap-readtext}
\end{figure}

\begin{figure}[H]
    \centering
    \includegraphics[width=0.9\textwidth]{fig_heatmap_spontaneous.png}
    \caption{Feature correlation heatmap (Spontaneous Dialogue task)}
    \label{fig:heatmap-spontaneous}
\end{figure}

\section{Dataset B: PD Speech Features}

\subsection{Source}

Pre-extracted acoustic features from Kaggle, containing 752 features per sample.

\subsection{Class Distribution}

\begin{table}[H]
\centering
\begin{tabular}{@{}lcc@{}}
\toprule
\textbf{Class} & \textbf{Samples} & \textbf{Percentage} \\
\midrule
HC (0) & 192 & 25.4\% \\
PD (1) & 564 & 74.6\% \\
\bottomrule
\end{tabular}
\caption{Class distribution in Dataset B}
\label{tab:dataset-b-distribution}
\end{table}

\textbf{Imbalance Ratio:} Severe ($\sim$25:75), necessitating class weighting.

\subsection{Feature Categories}

The 752 features span multiple acoustic domains:

\begin{table}[H]
\centering
\begin{tabular}{@{}lcl@{}}
\toprule
\textbf{Category} & \textbf{Count} & \textbf{Description} \\
\midrule
Baseline Features & 22 & Jitter, shimmer, HNR variants \\
Intensity & 3 & Intensity statistics \\
Formants & 36 & $F_1$--$F_4$ bandwidth features \\
MFCCs & 84 & MFCC coefficients \\
Wavelet & 182 & Wavelet decomposition features \\
TQWT & 432 & Tunable Q-factor features \\
\bottomrule
\end{tabular}
\caption{Feature categories in Dataset B}
\label{tab:dataset-b-features}
\end{table}

\subsection{Important Caveat}

\begin{quote}
\textbf{Warning:} No subject identifiers are available in Dataset B. Results may be optimistic due to potential subject overlap across cross-validation folds. The absence of subject identifiers prevents validation of true out-of-subject generalization.
\end{quote}

\section{Dataset Comparison}

\subsection{Key Differences}

\begin{enumerate}
    \item \textbf{Data format:} Raw audio vs.\ pre-extracted features
    \item \textbf{Sample size:} 37 subjects vs.\ 756 samples
    \item \textbf{Subject tracking:} Available vs.\ unavailable
    \item \textbf{Feature dimensionality:} 47--78 (extracted) vs.\ 752 (provided)
\end{enumerate}

\subsection{Implications for Evaluation}

\begin{itemize}
    \item Dataset A enables \textbf{grouped cross-validation} (more conservative, realistic estimates)
    \item Dataset B requires \textbf{standard cross-validation} (potentially optimistic estimates)
    \item Direct comparison is confounded by these methodological differences
\end{itemize}
