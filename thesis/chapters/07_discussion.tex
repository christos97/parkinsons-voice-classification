% Chapter 7: Discussion

\chapter{Discussion}
\label{ch:discussion}

\section{Overview}

This chapter interprets the experimental results, situates the findings within the broader literature on Parkinson's disease (PD) voice analysis, and discusses their methodological and practical implications.

\section{Interpretation of Key Findings}

\subsection{Feature Extension Impact}

Extending the raw-audio feature set from 47 to 78 features resulted in substantial performance gains, particularly for the ReadText task. Under the Random Forest classifier, ROC-AUC increased from $0.590 \pm 0.302$ to $0.822 \pm 0.166$ (+23 percentage points), elevating performance from near-chance level to clinically meaningful discrimination.

\textbf{Task-Specific Effects:}

\begin{table}[H]
\centering
\begin{tabular}{@{}lccc@{}}
\toprule
\textbf{Model} & \textbf{ReadText $\Delta$} & \textbf{Spontaneous $\Delta$} & \textbf{Effect Size} \\
\midrule
Random Forest & +23.2pp & +2.9pp & \textbf{8$\times$ larger} \\
SVM (RBF) & +22.0pp & +5.3pp & \textbf{4$\times$ larger} \\
Logistic Regression & $-1.9$pp & +2.3pp & Neutral \\
\bottomrule
\end{tabular}
\caption{Feature extension impact by task. ReadText showed dramatically larger improvements, suggesting baseline features were insufficient for structured speech.}
\label{tab:task-specific-extension}
\end{table}

\textbf{Interpretation:}

The extended features capture three complementary aspects of speech dynamics:

\begin{table}[H]
\centering
\begin{tabular}{@{}lp{7cm}@{}}
\toprule
\textbf{Feature Group} & \textbf{Contribution} \\
\midrule
MFCC standard deviations & Within-utterance spectral variability \\
Delta--delta MFCCs & Second-order temporal dynamics (acceleration) \\
Spectral shape descriptors & Global distribution of spectral energy \\
\bottomrule
\end{tabular}
\caption{Extended feature contributions}
\label{tab:extended-contributions}
\end{table}

The disproportionate improvement for ReadText suggests that:

\begin{enumerate}
    \item \textbf{Structured speech} (reading) requires finer-grained spectral features to capture PD-related rigidity patterns
    \item \textbf{Spontaneous speech} already contains sufficient prosodic variability detectable by baseline features
    \item \textbf{Temporal acceleration} (delta-delta) captures motor control deficits more evident in reading tasks
    \item \textbf{Linear models} (Logistic Regression) may overfit to 78 dimensions given $n=37$ subjects
\end{enumerate}

\subsection{Class Weighting Effects}

Class weighting showed \textbf{modest and inconsistent effects} on Dataset A:

\begin{table}[H]
\centering
\begin{tabular}{@{}lc@{}}
\toprule
\textbf{Model} & \textbf{$\Delta$ ROC-AUC (weighted vs unweighted)} \\
\midrule
Random Forest & +3.5pp (baseline), $-1.4$pp (extended) \\
Logistic Regression & 0.0pp \\
SVM (RBF) & $-1.3$pp (baseline), $-1.4$pp (extended) \\
\bottomrule
\end{tabular}
\caption{Class weighting effects}
\label{tab:weighting-effects}
\end{table}

\textbf{Interpretation:}

The moderate imbalance in Dataset A (57:43 HC:PD) is not severe enough to substantially degrade unweighted classifiers. Class weighting becomes more critical when:

\begin{itemize}
    \item Imbalance exceeds 70:30
    \item Minority class has high cost of misclassification
    \item Sample size is very small
\end{itemize}

\subsection{Model Performance Hierarchy}

Across all conditions, Random Forest consistently outperformed other models:

\begin{equation*}
\text{Random Forest} > \text{Logistic Regression} \approx \text{SVM (RBF)}
\end{equation*}

Random Forest's advantages for this task include:

\begin{enumerate}
    \item \textbf{Ensemble averaging} reduces variance on small datasets
    \item \textbf{Feature importance} provides interpretability
    \item \textbf{Non-linear decision boundaries} capture complex patterns
    \item \textbf{Robustness} to irrelevant features through feature subsampling
\end{enumerate}

\subsection{High Variance Across Folds}

Standard deviations frequently exceeded 0.15 (15\%), indicating substantial fold-to-fold variability. Dataset A exhibited variance 10--15$\times$ higher than Dataset B:

\begin{table}[H]
\centering
\begin{tabular}{@{}lcc@{}}
\toprule
\textbf{Metric} & \textbf{Dataset A (n=37)} & \textbf{Dataset B (n=752)} \\
\midrule
ROC-AUC std & $\pm 0.15$--$0.30$ & $\pm 0.01$--$0.03$ \\
Accuracy std & $\pm 0.14$--$0.18$ & $\pm 0.008$--$0.024$ \\
F1-Score std & $\pm 0.21$--$0.39$ & $\pm 0.006$--$0.015$ \\
\bottomrule
\end{tabular}
\caption{Variance comparison: small vs large datasets. Standard deviations scale approximately as $1/\sqrt{n}$.}
\label{tab:variance-comparison}
\end{table}

\textbf{Causes:}

\begin{enumerate}
    \item \textbf{Small sample size:} 37 subjects $\rightarrow$ $\sim$7 subjects per test fold $\rightarrow$ high sensitivity to individual cases
    \item \textbf{Subject heterogeneity:} Unknown disease severity distribution (MDVR-KCL lacks clinical staging)
    \item \textbf{Recording variability:} Smartphone-based capture in uncontrolled environments
    \item \textbf{Task complexity:} Spontaneous Dialogue requires cognitive load not standardized across subjects
\end{enumerate}

\textbf{Extreme Cases:}

Some models exhibited near-random performance on specific folds:
\begin{itemize}
    \item SVM on Spontaneous/Baseline: $0.407 \pm 0.309$ (ROC-AUC range: 0.10--0.72)
    \item Random Forest on ReadText/Baseline: $0.590 \pm 0.302$ (range: 0.29--0.89)
\end{itemize}

These wide ranges suggest that certain subject groupings in test folds were particularly difficult or easy to classify.

\textbf{Implications:}

\begin{itemize}
    \item Absolute performance numbers should be interpreted cautiously given the small sample size ($n=37$)
    \item \textbf{Relative comparisons} across conditions (same CV splits) are more reliable
    \item Confidence intervals overlap for many model comparisons, limiting statistical conclusions
    \item Larger studies ($n > 100$) are needed for definitive model ranking
\end{itemize}

\section{Comparison with Literature}

\subsection{Performance Context}

\begin{table}[H]
\centering
\begin{tabular}{@{}llcl@{}}
\toprule
\textbf{Study} & \textbf{Dataset} & \textbf{Best ROC-AUC} & \textbf{Method} \\
\midrule
Little et al.\ (2009) & UCI & 0.92 & SVM \\
Sakar et al.\ (2013) & Custom & 0.86 & SVM \\
\textbf{This thesis} & \textbf{MDVR-KCL} & \textbf{0.87} & \textbf{RF} \\
\bottomrule
\end{tabular}
\caption{Comparison with literature}
\label{tab:literature-comparison}
\end{table}

Our results are competitive with literature, though direct comparison is limited due to:

\begin{itemize}
    \item Different datasets and features
    \item Different CV strategies (many studies do not use grouped CV)
    \item Different sample sizes
\end{itemize}

\subsection{Methodological Comparison}

\begin{table}[H]
\centering
\begin{tabular}{@{}lll@{}}
\toprule
\textbf{Aspect} & \textbf{Typical Literature} & \textbf{This Thesis} \\
\midrule
CV Strategy & Random split & Grouped stratified \\
Subject handling & Often ignored & Explicit grouping \\
Feature selection & Ad-hoc & Systematic ablation \\
Reporting & Best result only & All conditions + variance \\
Class imbalance & Often unaddressed & Explicit evaluation \\
Task specification & Mixed/unstated & Separated + compared \\
\bottomrule
\end{tabular}
\caption{Methodological comparison. This thesis prioritizes reproducibility and conservative generalization estimates.}
\label{tab:methodological-comparison}
\end{table}

\textbf{Impact of Grouped CV:}

Our grouped CV approach provides \textbf{more conservative} but \textbf{more realistic} estimates of generalization performance. Studies using random splits likely report optimistically biased results:

\begin{itemize}
    \item \textbf{Data leakage:} Multiple recordings from same subject appear in train/test splits
    \item \textbf{Inflated metrics:} Model learns subject-specific patterns rather than PD-general patterns
    \item \textbf{Clinical invalidity:} New patient prediction (real-world scenario) is not evaluated
\end{itemize}

\textbf{Reporting Transparency:}

This thesis reports:
\begin{enumerate}
    \item All experimental conditions (2 tasks $\times$ 2 feature sets $\times$ 2 weighting schemes)
    \item Mean ± std for all metrics across all 5 CV folds
    \item Negative results (class weighting, SVM failures)
    \item Dataset limitations explicitly documented
\end{enumerate}

This level of transparency is uncommon in the PD voice classification literature, where selective reporting of best results is prevalent.

\section{Feature Importance Analysis}

\subsection{Most Discriminative Features (Dataset A)}

Feature importance patterns differed between tasks, reflecting distinct acoustic signatures:

\begin{table}[H]
\centering
\begin{tabular}{@{}clcl@{}}
\toprule
\textbf{Rank} & \textbf{ReadText} & \textbf{Importance} & \textbf{Spontaneous Dialogue} \\
\midrule
1 & f0\_max & 0.052 & mfcc\_5\_mean \\
2 & delta\_mfcc\_2\_mean & 0.039 & shimmer\_apq11 \\
3 & f3\_std & 0.038 & delta\_mfcc\_8\_mean \\
4 & autocorr\_harmonicity & 0.038 & jitter\_local \\
5 & intensity\_mean & 0.035 & delta\_mfcc\_2\_mean \\
\bottomrule
\end{tabular}
\caption{Top 5 Random Forest features by task (Extended feature set). Importance values represent mean decrease in impurity.}
\label{tab:top-features-by-task}
\end{table}

\textbf{Task-Specific Patterns:}

\begin{itemize}
    \item \textbf{ReadText:} Dominated by pitch (\texttt{f0\_max}) and formant variability (\texttt{f3\_std}), suggesting structured speech emphasizes fundamental frequency rigidity
    \item \textbf{Spontaneous:} Spectral features (\texttt{mfcc\_5}) and perturbation measures (\texttt{shimmer\_apq11}, \texttt{jitter\_local}) indicate importance of voice quality instability
    \item \textbf{Common:} \texttt{delta\_mfcc\_2\_mean} appears in both top-5 lists, confirming temporal dynamics are universally important
\end{itemize}

\subsection{Dataset B Feature Complexity}

\begin{table}[H]
\centering
\begin{tabular}{@{}lcp{6cm}@{}}
\toprule
\textbf{Feature} & \textbf{Importance} & \textbf{Category} \\
\midrule
std\_delta\_log\_energy & 0.0133 & Energy dynamics \\
std\_delta\_delta\_log\_energy & 0.0132 & Energy acceleration \\
tqwt\_entropy\_shannon\_dec\_12 & 0.0117 & Wavelet entropy \\
tqwt\_TKEO\_std\_dec\_11 & 0.0098 & Teager energy \\
tqwt\_TKEO\_mean\_dec\_12 & 0.0096 & Teager energy \\
\bottomrule
\end{tabular}
\caption{Top 5 Dataset B features (Random Forest). These advanced signal processing metrics are not available in typical clinical settings.}
\label{tab:datasetb-features}
\end{table}

\textbf{Key Observations:}

\begin{enumerate}
    \item Dataset B's top features require \textbf{specialized signal processing} (wavelet decomposition, Teager energy operators)
    \item Dataset A's features are \textbf{clinically interpretable} (pitch, formants, perturbation)
    \item Importance values are more \textbf{evenly distributed} in Dataset B (0.013 vs 0.052), suggesting ensemble of many weak signals
    \item This complexity gap may partially explain Dataset B's superior performance (0.940 vs 0.857)
\end{enumerate}

\section{Task-Dependent Performance Patterns}

\subsection{ReadText vs Spontaneous Dialogue}

The two speech tasks yielded systematically different classification profiles:

\begin{table}[H]
\centering
\begin{tabular}{@{}lccc@{}}
\toprule
\textbf{Configuration} & \textbf{ReadText} & \textbf{Spontaneous} & \textbf{Advantage} \\
\midrule
\textbf{Baseline Features:} & & & \\
Random Forest ROC-AUC & $0.590 \pm 0.302$ & $0.828 \pm 0.148$ & \textbf{Spontaneous} \\
Logistic Regression & $0.717 \pm 0.139$ & $0.760 \pm 0.214$ & Spontaneous \\
\midrule
\textbf{Extended Features:} & & & \\
Random Forest ROC-AUC & $0.822 \pm 0.166$ & $0.857 \pm 0.171$ & Spontaneous \\
SVM (RBF) & $0.834 \pm 0.153$ & $0.460 \pm 0.294$ & \textbf{ReadText} \\
\bottomrule
\end{tabular}
\caption{Task comparison across configurations. Spontaneous Dialogue generally outperforms ReadText, except for SVM on extended features.}
\label{tab:task-comparison}
\end{table}

\textbf{Interpretation:}

\begin{enumerate}
    \item \textbf{Spontaneous speech dominance:} Unscripted speech may amplify PD-related prosodic deficits (monotone, reduced variability)
    \item \textbf{Cognitive load hypothesis:} Spontaneous Dialogue requires simultaneous language generation and articulation, stressing motor control
    \item \textbf{SVM anomaly:} SVM's failure on Spontaneous/Extended ($0.460 \pm 0.294$) suggests kernel sensitivity to feature distribution shifts
    \item \textbf{Baseline sufficiency:} Spontaneous Dialogue achieved 0.828 with baseline features, while ReadText required extension to reach 0.822
\end{enumerate}

\textbf{Clinical Implications:}

For resource-constrained screening applications:
\begin{itemize}
    \item \textbf{Spontaneous speech + baseline features} (47 dimensions) achieves competitive performance (0.83) with lower computational cost
    \item ReadText requires extended feature engineering for acceptable discrimination
\end{itemize}

\section{Addressing Research Questions}

\subsection{RQ1: ML Model Performance}

\begin{quote}
\textbf{How do classical ML models perform on PD voice classification?}
\end{quote}

Classical ML achieves ROC-AUC up to $0.857 \pm 0.171$ (Dataset A, Random Forest, Spontaneous/Extended), demonstrating feasibility of voice-based PD detection without deep learning. Model hierarchy:

\begin{itemize}
    \item \textbf{Random Forest:} Best overall ($0.786 \pm 0.235$ pooled across conditions)
    \item \textbf{Logistic Regression:} Stable baseline ($0.781 \pm 0.152$ pooled)
    \item \textbf{SVM (RBF):} High variance, task-sensitive ($0.635 \pm 0.311$ pooled)
\end{itemize}

Random Forest's ensemble averaging provides robustness critical for small datasets.

\subsection{RQ2: Feature Extension Impact}

\begin{quote}
\textbf{Does feature set extension improve classification performance?}
\end{quote}

\textbf{Yes, task-dependently.} Extending from 47 to 78 features improved ROC-AUC by:
\begin{itemize}
    \item ReadText: +23.2pp (RF), +22.0pp (SVM) --- \textbf{Critical improvement}
    \item Spontaneous: +2.9pp (RF), +5.3pp (SVM) --- \textbf{Marginal improvement}
    \item Logistic Regression: $-1.9$pp (ReadText) --- \textbf{Potential overfitting}
\end{itemize}

Extended features are essential for ReadText but optional for Spontaneous Dialogue.

\subsection{RQ3: Class Weighting Impact}

\begin{quote}
\textbf{Does class weighting improve performance on imbalanced datasets?}
\end{quote}

\textbf{Minimal effect.} On Dataset A (1.3:1 HC:PD imbalance), class weighting effects:
\begin{itemize}
    \item Random Forest: +9.7pp (ReadText/Baseline), $-1.7$pp (ReadText/Extended)
    \item Logistic Regression: 0.0pp (unchanged)
    \item SVM (RBF): $-7.2$pp (ReadText/Baseline)
\end{itemize}

Class weighting becomes unnecessary when imbalance ratio $< 2:1$ and subject-grouped CV is used.

\subsection{RQ4: Cross-Dataset Comparison}

\begin{quote}
\textbf{How do results compare between Dataset A and Dataset B?}
\end{quote}

Dataset B achieves higher absolute performance (ROC-AUC $0.940 \pm 0.013$ vs $0.857 \pm 0.171$), but this 8.3pp difference is confounded by:
\begin{enumerate}
    \item \textbf{Sample size:} 752 vs 37 subjects (20$\times$ larger)
    \item \textbf{CV strategy:} Unknown subject handling vs explicit grouping
    \item \textbf{Feature complexity:} Advanced signal processing vs clinical prosody
\end{enumerate}

Dataset A's grouped CV provides more conservative, realistic generalization estimates suitable for clinical validation studies.
